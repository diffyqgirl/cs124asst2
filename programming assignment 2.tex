\documentclass{article}
\begin{document}
\title{CS 124 Programming Assignment 2}
\author{30943147}
\maketitle
\section{Caching}
When multiplying two matrices $A*B=C$ using the naive matrix multiplication method, the elements of C are given as follows
$$C[i][j] = \sum\limits_k A[i][k]*B[k][j]$$
This is naturally implemented with 3 nested for loops iterating over $i$, $j$, and $k$. There are $3!=6$ possible permutations of their order. I tested all of them experimentally to determine which was best. Runtime measurements were taken by running the naive algorithm on $n \times n$ matrices with randomly generated entries for $n=1200$. 
\begin{center}
\begin{tabular}{ |c|c| } 
 \hline
 ordering & runtime\\ 
\hline
 i, j, k &  5.81s \\ 
\hline
 i, k, j & 1.73s \\ 
\hline
j, i, k & \\
\hline
j, k, i & \\
\hline
k, i, j & \\
k, j, i & \\
 \hline
\end{tabular}
\end{center}
\end{document}